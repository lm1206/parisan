

\documentclass{ctexart} % Default font size is 12pt, it can be changed here
\ctexset{section/format = {\Large\bfseries}}
\usepackage{geometry} % Required to change the page size to A4
\geometry{a4paper} % Set the page size to be A4 as opposed to the default US Letter
\geometry{left=1.6cm,right=1.6cm,top=3.5cm,bottom=3.5cm}
%\usepacjage{subfigure}
\usepackage{graphicx} % Required for including pictures
\usepackage{amsmath}
\usepackage{amssymb,bm}
%\usepackage{abstract}
\usepackage{float} % Allows putting an [H] in \begin{figure} to specify the exact location of the figure
%\usepacjage{wrapfig} % Allows in-line images such as the example fish picture

%\linespread{1.2} % Line spacing


%\setlength\parindent{0pt} % Uncomment to remove all indentation from paragraphs
\usepackage{enumerate}
\graphicspath{{Pictures/}} % Specifies the directory where pictures are stored

 
\newtheorem{definition}{\hspace{2em}定义}[section] % 如果没有章, 只有节, 把上面的[chapter]改成[section] 
\newtheorem{theorem}[definition]{\hspace{2em}定理} 
\newtheorem{axiom}[definition]{\hspace{2em}公理} 
\newtheorem{lemma}[definition]{\hspace{2em}引理} 
\newtheorem{proposition}[definition]{\hspace{2em}命题} 
\newtheorem{corollary}[definition]{\hspace{2em}推论} 
\newtheorem{remark}{\hspace{2em}注}[section] %类似地定义其他“题头”. 这里“注”的编号与定义、定理等是分开的




\begin{document}

%----------------------------------------------------------------------------------------
%	TITLE PAGE
%----------------------------------------------------------------------------------------

\begin{titlepage}

\newcommand{\HRule}{\rule{\linewidth}{0.5mm}} % Defines a new command for the horizontal lines, change thicjness here

\center % Center everything on the page
%%\includegraphics[width=2.5cm]{Pictures/SCE.png}
%%\textsc{\LARGE University Name}\\[1.5cm] % Name of your university/college
%%\textsc{\Large Maior Heading}\\[0.5cm] % Maior heading such as course name
%%\textsc{\large Minor Heading}\\[0.5cm] % Minor heading such as course title

\HRule \\[0.4cm]
{ \huge \bfseries Pricing Parisian options under a Gaussian process}\\[0.4cm] % Title of your document
\HRule \\[1.5cm]

\begin{minipage}{0.4\textwidth}
\begin{flushleft} \large
\emph{Author:}\\
Ming \textsc{Li} % Your name
\end{flushleft}
\end{minipage}
\begin{minipage}{0.4\textwidth}
\begin{flushright} \large
\emph{Department:} \\
Mathematics
\end{flushright}
\end{minipage}\\[4cm]

{\large \today}\\[3cm] % Date, change the \today to a set date if you want to be precise

%\includegraphics{Logo}\\[1cm] % Include a department/university logo - this will require the graphicx pacjage

\vfill % Fill the rest of the page with whitespace
\end{titlepage}
\newpage
\begin{abstract}
This study extends the Parisian option pricing model to general Gaussian processes to adapt more asset pricing model assumptions. This study uses a Wick-Ito integral to derive the consecutive and cumulative Parisian option pricing PDEs under Gaussian processes and construct an unconditionally stable implicit difference scheme. Additionally, a numerical method with efficient computation, namely, a chasing algorithm, is used to solve the pricing problem. When analysing the example of the price of consecutive and cumulative Parisian options under any Gaussian process, because the counter is not zero, the cumulative Parisian option is more likely a knock-out option, and the price is lower than that of the consecutive Parisian option. The variables that affect the prices of the Parisian option under a Gaussian process are a slow-growing variance function, smaller volatility, more likely profit and loss forecasts, steeper graph of the price of the Parisian option, and higher peak value.

{\bf Key Words: }Parisian Option, Gaussian Process, Chasing Algorithm, Brownian Motion, Implicit Difference Equations
\end{abstract}

\section{Introduction}
A Parisian option is a complex, path-dependent option that can be classified as cumulative or consecutive depending on the duration time. A consecutive Parisian option is the price of an underlying asset with a required barrier level that is continuous and can remain for some time; it can knock-in or knock-out above or below the barrier. A cumulative Parisian option is an asset price that needs to accumulate above or below the barrier level for some time to knock-in or knock-out with an optional maturity time as the limit. For every knock-in or knock-out time, it continues to accumulate after the pre-setting time. The pre-setting time determines whether the option is triggered. A cumulative Parisian option is more likely to trigger the barrier level than a consecutive Parisian option, depending on the option conditions. In the knock-out option, the price of a consecutive Parisian option is higher than that of the cumulative Parisian option; it is opposite for the knock-in option. The Parisian option is mainly used in the foreign exchange market and can have hybrid derivative pricing. For example, the foreign exchange Parisian option can reduce the impact of the exchange rate; it changes option prices to avoid market speculation. In a convertible bond, conversion and redemption clauses contain Parisian option features.

The main methods for pricing Parisian options are the probability method and the partial differential equation (PDE) method. Under the framework of Black-Scholes option pricing, Haber, Schonbucher and Wilmott (1999) derived the PDE for a Parisian option using stock price, duration time and terminal time. These three variables were used with an explicit difference method to solve the problem; however, this numerical method converges slowly, and its stability is not good. Anderluha (2008) fixed the boundary and terminal conditions to find the Parisian option PDE using a hitting time simulation to price consecutive and cumulative Parisian options. Kowk and Lau (2001) priced consecutive and cumulative Parisian options by using a forward-shooting grid under a trinomial tree framework. Longstaff and Schwartz (2001) used a least-square Monte Carlo method to price the American-style Parisian option.\\

These models assume that the Parisian option’s underlying asset price follows geometric Brownian motion. However, Dai and Singleton (2000), via an empirical study of the financial markets, suggest that the change in financial asset price is not a random walk but has varying degrees of long-range dependence and auto-correlation. A financial asset logarithm yields a fat-tail distribution rather than a normal distribution. Peters (1989) noted that the fractional market hypothesis uses an R/S algorithm to detect differences in financial markets and shows that fractional and non-cyclical structures exist in financial markets; these characteristics are significantly different from geometric Brownian motion. Instead, fractional, bi-fractional and sub-fractional Brownian motion describe long-range dependence and auto-correlation very well and yield a fat-tail distribution (Xiao et al., 2012; Lei and Nualart, 2009; Yan, Shen and He, 2011). \\

This paper analyses the PDE of consecutive and cumulative Parisian options under various Gaussian processes. First, by introducing the Wick-Ito integral, we construct the PDE for a financial derivative under a complete financial market with no arbitrage opportunity. Then, we set up the boundary and initial conditions for consecutive and cumulative Parisian options under a general Gaussian process. To solve the PDE, we construct numerical difference equations and introduce the chasing algorithm to reduce the complexity of the calculations. Finally, we examine a case study in which we introduce various Gaussian processes into our model and analyse the influence of the parameters.


\section{Gaussian Process}
The financial market is a complex system, and most investors do not have access to adequate information. Investors gain profits while taking risks. In the efficient market hypothesis, noise is defined as the deviation between the price and the value of financial assets. Most investors are noise traders: they normally treat noise information as factual information to trade. Information traders use noise-trader-characteristics to make profits due to a herd mentality. This introduces market excess and volatility, which leads to market volatility clustering. Similar yields on different time scales with similar distributions of frequency, such as auto-correlation in time scales, results from new investors who take the same risk level as a crowd of investors. Financial markets have a long memory, which includes past information that not only affects current markets but also affects future markets.\\

A Gaussian process is determined by its mean and covariance functions. Suppose that $\{W_t\}_{t\geq 0}$ is a Gaussian process on the probability space $(\Omega,F,P)$with independent increments,$W_0=0$, that satisfies the following properties
%\begin{enumerate}[(1)]
%\item $E(W_t)=0$,$t\geq 0$ is a mean function
%\item $E(W_t,W_s)=F(t,s)<\infty$,$t,s\geq 0$ is the covariance %function
%\end{enumerate}

If $F(t,s)=t\wedge s$, then $W_t$ is a standard Brownian motion. If $F(t,s)=1/2(t^{2H}+s^{2H}-|t-s|^{2H})$, then $W_t$  is a fractional Brownian motion with Hurst number H. When $H\in(0,1/2)$, increments are negatively correlated, $W_t$has a short-term dependence and anti-persistence, and its future growth does not follow past trends. When $H\in(1/2,1)$, $W_t$  has a long-term dependence and persistence and future trends will be extended by past growth trends.\\

The covariance function of a bi-fractional process is F$F(t,s)=1/2^K[(t^{2H}+s^{2H})^K-|t-s|^{2HK}]$. It has properties such as auto-correlation and long-memory of fractional process, and when $2HK=1$, $W_t$ is a semi-martingale.\\
The sub-fractional process also has auto-correlation and long-memory properties, but it has a faster degradation speed than does fractional Brownian motion. Its covariance function is $F(t,s)=t^{2H}+s^{2H}-1/2(|t+s|^{2H}+|t-s|^{2H})$.\\

Mixed fractional Brownian motion is a linear combination of standard Brownian motion and fractional Brownian motion. Suppose$B_t$ is a Brownian motion and $B_t^H$ is a fractional Brownian motion; then, $W_t=\alpha B_t+\beta B_t^H$, where $\alpha,\beta \in R$ and its covariance function is  $F(t,s)=\alpha^2(t\wedge s)+\beta(t^{2H}+s^{2H}-|t-s|^{2H})/2$ 
We express $F(t,t)=F_t$,$dF_t/dt=U_t$ and provide a summary of each motion in Table \ref{tabl1}:

\begin{table}[htbp]
\centering
\caption{各种高斯过程的协方差函数}
\label{tabl1}
\begin{tabular}{ccc}
\hline
高斯过程& $F_t$ & $U_t$ \\
\hline
布朗运动 & $t$ & $1$ \\
分数布朗运动 & $t^{2H}$ & $2Ht^{2H-1}$ \\
双分数布朗运动 & $t^{2HK}$ & $2HKt^{2HK-1}$ \\
次分数布朗运动 & $(2-2^{2H-1})t^{2H}$ & $(4-2^{2H})Ht^{2H-1}$ \\
混合分数布朗运动 & $\alpha^2t+\beta^2t^{2H}$ & $\alpha^2+2H\beta^2t^{2H-1}$ \\
\hline
\end{tabular}
\end{table}

本文主要研究在一般的高斯过程下连续型和累计型巴黎期权的偏微分方程,并通过构造差分方程进行求解。最后选择不同$U_t$的高斯过程,通过算例求解,讨论各个模型参数对连续型和累计型巴黎期权价格的影响。以及两种类型巴黎期权的比较。


\section{Wick-Ito 积分}
当高斯过程$W_t$既不是Markovian过程,也不是半鞅时,Rogers证明了按照路径型积分建立的金融市场数学模型存在套利机会。而在一个存在套利机会的市场中对金融衍生品定价是不切实际的。因此,本文使用如下的Wick-Ito积分
\begin{equation}
\int_a^bf(t,\omega)dW_t=\lim_{\| \Pi\| \rightarrow0}\sum^{n-1}_{k=0}f(t_k,\omega)\diamond (W_{t+1}-W_t)
\end{equation}
其中$\diamond$代表Wick乘积。由此定义的随机积分宽假下,金融市场数学模型是无套利的并且市场是完备的。

下面引入关于高斯过程的Wick-Ito公式,其证明过程请参考文献\cite{ref16}
\begin{lemma}
\label{x1}
设$Y_t$是一方差有界的中心型高斯变量,记$F_t=E(Y^2_t)$,函数$f(t,y):[0,+\infty]\times R\rightarrow R$是连续的,其偏导数$\frac{\partial f}{\partial t}$,$\frac{\partial f}{\partial y}$,$\frac{\partial^2 f}{\partial y^2}$均存在且满足指数型增长条件,即$\forall t\in[0,T]$,$x \in R$有
\begin{equation}
|\frac{\partial^k f(t,y)}{\partial y^k}|\leq C_T\exp(c_Ty^2),\quad k=0,1,2
\end{equation}
则Wick-Ito公式的积分形式为
\begin{equation}
f(T,Y_T)=f(0,Y_0)+\int^T_0\frac{\partial f}{\partial y}(t,Y_t)dt+\int^T_0\frac{\partial f}{\partial y}(t,Y_t)\diamond dY_t+\frac{1}{2}\frac{\partial^2 f}{\partial y^2}(t,Y_t)dF_t
\end{equation}
其中,$C_T$,$c_T$均大于零的常数,且满足$c_T<\frac{1}{4}\sup_{t\in[0,T]}F_t^{-1}$,$\diamond$表示Wick乘积,同时Wick-Ito公式的微分形式为
\begin{equation}
df(T,Y_T)=f(0,Y_0)+\int^T_0\frac{\partial f}{\partial y}(t,Y_t)dt+\frac{\partial f}{\partial y}(t,Y_t)\diamond dY_t+\frac{1}{2}\frac{\partial^2 f}{\partial y^2}(t,Y_t)dF_t
\end{equation}
\end{lemma}

由引理\ref{x1},我们可以得到以下定理\cite{ref16}
\begin{theorem}
设$W_t$为一方差有界的高斯过程,标的资产$S_t$为一随机过程且满足
\begin{equation}
dS_t=\mu S_tdt+\sigma S_t\diamond dW_t
\end{equation}
$V(S_t,t)$为一关于$S_t$的金融衍生品的价格,$r$为无风险利率,$F(t)=E(W^2_t)$,$U_t=\frac{dF(t)}{dt}$则任意时刻$t\in[0,T]$,$V(S_t,t)$满足偏微分方程
\begin{equation}
\label{f1}
\frac{\partial V}{\partial t}+\frac{1}{2}\sigma^2S^2U_t\frac{\partial V}{\partial S^2}+\mu S\frac{\partial V}{\partial t}=rV
\end{equation}


\end{theorem}


\section{高斯运动下巴黎期权定价模型}
设$S(t)$为一个公司股票的价格过程,满足下列随机微分方程
\begin{equation}
dS_t=(r-q)S_t dt+\sigma S_t \diamond dW_t
\end{equation}
其中$r$为无风险利率,$\sigma^2$为股票价格波动率方差。$W_t$为一高斯过程,$F_t=E(W^2_t)$,$dF_t/dt=U_t$。
\subsection{连续型向上敲出看涨巴黎期权}
考虑向上敲出看涨欧式巴黎期权(Up-and-Out Call)的定价问题。$J$表示股票价格$S_t$在障碍水平$L$之上连续时间的长度,定义为
\begin{equation}
J(t)=t-\sup(t'\leq t|S_t'\leq D)
\end{equation}
于是有
\begin{equation}
dJ(t)=\left\{
\begin{aligned}
&dt,\quad &S_t>L \\
&-J(t_-),\quad&S_t=L\\
&0,\quad&S_t<L&
\end{aligned}
\right.
\end{equation}
巴黎期权的价格$V$是标的资产价格$S$,时间$t$及资产价格连续超过障碍的时间当前计数值$J$的函数,记为$V(S,J,t)$。假定障碍价格$L>E$,根据动态复制的理论,结合方程\ref{f1},向上敲出看涨的巴黎期权的价格应满足方程
\begin{equation}
\label{eq}
\frac{\partial V}{\partial t}+H(S-L)\frac{\partial V}{\partial J}+(r-q)S\frac{\partial V}{\partial S}+\frac{1}{2}\sigma^2S^2U\frac{\partial^2 V}{\partial S^2}-rV=0
\end{equation}
求解区间为$0\leq J \leq D$,$0\leq t \leq T$,$0< S < \infty$,其中函数
\begin{equation}
H(x)=\left\{
\begin{aligned}
1 \quad x>0 \\
0 \quad x\leq 0
\end{aligned}
\right.
\end{equation}
根据巴黎期权的定义,当标的资产价格连续超过障碍$L$的时间达到$D$时(即$J=D$时)即敲出,其价值为零。
\begin{equation}
V(S,t,D)=0
\end{equation}
每当发生$S<L$时,$J$重置为零。当$J=0$时,考虑$V$在$J=0$处的连续性,定解条件为
\begin{equation}
\label{bc1}
V(L,t,J)=V(L,t,0)
\end{equation}
当$t=T$时,期权没有敲出,执行期权的终端支付,即满足
\begin{equation}
V(S,T,J)=(S-E)^{+} 
\end{equation}
由于当$S<L$时,$J$恒为$0$,因此可以简记为$V(S,t)$,满足
\begin{equation}
\label{down}
\frac{\partial V}{\partial t}+(r-q)S\frac{\partial V}{\partial S}+\frac{1}{2}\sigma^2S^2U_t\frac{\partial^2 V}{\partial S^2}-rV=0
\end{equation}
当$S>L$,$0<J<D$时,$V(S,t,J)$满足
\begin{equation}
\label{up}
\frac{\partial V}{\partial t}+\frac{\partial V}{\partial J}+(r-q)S\frac{\partial V}{\partial S}+\frac{1}{2}\sigma^2S^2U_t\frac{\partial^2 V}{\partial S^2}-rV=0
\end{equation}
\subsection{累计型向上敲出看涨巴黎期权}
在累计时间下,在$S<L$时,计时器$J$不清零也不计时,其微分表达式为
\begin{equation}
dJ(t)=\left\{
\begin{aligned}
&dt,\quad &S_t\geq L \\
&0,\quad&S_t<L&
\end{aligned}
\right.
\end{equation}
累计型向上敲出看涨巴黎期权的价值同样满足方程\eqref{eq},但是由于$J$不重置,其定解条件不包含\eqref{bc1}
\begin{equation}
\label{bc2}
\left\{
\begin{aligned}
V(S,T,J)&=(S-E)^{+} \\
V(S,t,D)&=0
\end{aligned}
\right.
\end{equation}

\section{数值求解}
离散化方程\eqref{down}与\eqref{up},令$S_i=S_0+i\Delta h$,$i\in[0,M]$,$S_{I}=L$;$t_n=n\Delta t$,$n\in[0,N]$,$N\Delta t=T$;$J_j=j\Delta t,j\in[0,K]$,$K\Delta t=D$。

当$S<L$时,构造差分格式
\begin{equation}
\frac{V^{n+1}_i-V^n_i}{\Delta t}+(r-q)S_i\frac{V^{n}_{i+1}-V^{n}_{i-1}}{2\Delta h}+\frac{1}{2}\sigma^2S_i^2U_{n}\frac{V_{i+1}^{n}-2V_i^{n}+V_{i-1}^{n}}{\Delta h^2}-rV^{n}_i=0
\end{equation}
即
\begin{equation}
\label{hcon1}
V^{n}_i-(a^n_iV^{n}_{i-1}+b^n_iV^{n}_i+c^n_iV^{n}_{i+1})=V^{n+1}_i
\end{equation}
其中
$$
a^n_i=-\frac{(r-q)S_i\Delta t}{2\Delta h}+\frac{\sigma^2S_i^2U_{n}\Delta t}{2\Delta h^2},\quad b^n_i=-r\Delta t-\frac{\sigma^2S_i^2U_{n}\Delta t}{\Delta h^2},\quad c^n_i=\frac{(r-q)S_i\Delta t}{2\Delta h}+\frac{\sigma^2S_i^2U_{n}\Delta t}{2\Delta h^2}
$$

当$S>L$时,构造差分格式
\begin{equation}
\frac{V^{n+1}_{i,j+1}-V^{n}_{i,j}}{\Delta t}+(r-q)S_i\frac{V^{n}_{i+1,j}-V^{n}_{i-1,j}}{2\Delta h}+\frac{1}{2}\sigma^2S_i^2U_{n}\frac{V_{i+1,j}^{n}-2V_{i,j}^{n}+V_{i-1,j}^{n}}{\Delta h^2}-rV^{n}_{i,j}=0
\end{equation}
即
\begin{equation}
\label{hcon2}
V^n_{i,j}-(a^n_iV^{n}_{i-1,j}+b^n_iV^{n}_{i,j}+c^n_iV^{n}_{i+1,j})=V^{n+1}_{i,j+1}
\end{equation}

当$S=L$时,综合\eqref{hcon1}和\eqref{hcon2}以及定解条件\eqref{bc1},需满足
\begin{equation}
V^n_{I}-(a^n_IV^{n}_{I-1}+b^n_IV^{n}_{I}+c^n_IV^{n}_{I+1,0})=V^{n+1}_{I}
\end{equation}
\begin{equation}
V^{n}_{I+1,0}-(a^n_IV^{n}_{I}+b^n_IV^{n}_{I+1,0}+c^n_IV^{n}_{I+2,0})=V^{n+1}_{I+1,1}
\end{equation}

在两端,使用如下线性近似
\begin{equation}
(1-2a^n_0-b^n_0)V^n_0+(a^n_0-c^n_0)V^{n}_{1}=V^{n+1}_0
\end{equation}
\begin{equation}
(1-2c^n_M-b^n_M)V^n_{M,j}+(c^n_M-a^n_M)V^{n}_{M-1,j}=V^{n+1}_{M,j+1}
\end{equation}

终端函数为
\begin{equation}
\label{hb}
V_{i,j}^N=(S_i-E)^+,\quad V_{i,K}^n=0
\end{equation}

方程\eqref{hcon1}和\eqref{hcon2}可写成矩阵形式
\begin{equation}
\label{meq1}
\left(
\begin{matrix}
1-2a^n_0-b^n_0 & a^n_0-c^n_0 & 0       &\ldots  & \ldots &\ldots     & 0 \\
-a^n_1       & 1-b^n_1   & -c^n_1    &\ldots  & \ldots &\ldots     & 0       \\
\vdots     & \ddots  & \ddots  &\ddots  & \vdots & \vdots   & \vdots   \\
0          & \ldots  & -a^n_I    & 1-b^n_I  & -c^n_I    & \ldots  & 0    \\
\vdots     & \vdots  & \vdots  &\ddots  & \ddots & \ddots   & \vdots   \\
0          & \ldots  & \ldots  &\ldots  & a^n_{M-1} & 1-b^n_{M-1} & -c^n_{M-1}   \\
0          & \ldots  & \ldots  &\ldots  &0        & c^n_M-a^n_M & 1-2c^n_M-b^n_M      
\end{matrix}
\right)
\left(
\begin{matrix}
V_0^n\\
V_1^n\\
\vdots\\
V_I^n\\
\vdots\\
V_{M-1,0}^n\\
V_{M,0}^n
\end{matrix}
\right)
=
\left(
\begin{matrix}
V_0^{n+1}\\
V_1^{n+1}\\
\vdots\\
V_I^{n+1}\\
\vdots\\
V_{M-1,1}^{n+1}\\
V_{M,1}^{n+1}
\end{matrix}
\right)
\end{equation}
以及
\begin{equation}
\label{meq2}
\left(
\begin{matrix}
1          & 0       & 0       &\ldots  & \ldots &\ldots     & 0 \\
-a^n_I       & 1-b^n_I   & -c^n_I    &\ldots  & \ldots &\ldots     & 0       \\
\vdots     & \ddots  & \ddots  &\ddots  & \vdots & \vdots    & \vdots   \\
0          & \ldots  & \ldots  &\ldots  & a^n_{M-1} & 1-b^n_{M-1} & -c^n_{M-1}   \\
0          & \ldots  & \ldots  &\ldots  &0        & c^n_M-a^n_M   & 1-2c^n_M-b^n_M      
\end{matrix}
\right)
\left(
\begin{matrix}
V_I^n\\
V_{I+1,j}^{n}\\
\vdots\\
V_{M-1,j}^{n}\\
V_{M,j}^{n}
\end{matrix}
\right)
=
\left(
\begin{matrix}
V_I^n\\
V_{I+1,j+1}^{n+1}\\
\vdots\\
V_{M-1,j+1}^{n+1}\\
V_{M,j+1}^{n+1}
\end{matrix}
\right)
\end{equation}

求解类似于\eqref{meq1}与\eqref{meq2}的系数矩阵为对角占优的三对角线方程组,常常使用的方法为追赶法,追赶法的计算量比Guess消去法的计算量要小的多。设方程组$Ax=f$为以下形式
\begin{equation}
\left(
\begin{matrix}
b_1       & c_1   &  & &\\
a_1       & b_1   & c_1    & & \\
          & \ddots  & \ddots  &\ddots   & \\
          &        & a_{n-1}  &b_{n-1}  & c_{n-1}   \\
          &        &          &a_n      &b_n
\end{matrix}
\right)
\left(
\begin{matrix}
x_1\\
x_2\\
\vdots\\
x_{n-1}\\
x_n^{n}
\end{matrix}
\right)
=
\left(
\begin{matrix}
f_1\\
f_2\\
\vdots\\
f_{n-1}\\
f_n
\end{matrix}
\right)
\end{equation}
求解$Ax=f$等价于解两个三角形方程组:
\begin{itemize}
\item 计算$\{\beta_i\}$:
\begin{equation}
\beta_1=c_1/b_1,\quad \beta_i=c_i/(b_i-\alpha_i\beta_i),\quad i=2,3,\ldots,n-1
\end{equation}

\item $Ly=f$,求$y$:
\begin{equation}
y_1=f_1/b_1,\quad y_i=(f_i-a_iy_{i-1})/(b_i-a_i\beta_{i-1}),\quad i=2,3,\ldots,n
\end{equation}

\item $Ux=y$,求$x$:
\begin{equation}
x_n=y_n,\quad x_i=y_i-\beta_ix_{i+1},\quad i=n-1,n-2,\vdots,1
\end{equation}

\end{itemize}

当$V(S,t,J)$为累计型巴黎期权时,由于累计型巴黎期权无需重置条件,故可使用差分方程
\begin{equation}
\label{hcmu}
\left\{
\begin{aligned}
V^n_{i,j}-(a^n_iV^{n}_{i-1,j}+b^n_iV^{n}_{i,j}+c^n_iV^{n}_{i+1,j})&=V^{n+1}_{i,j} \quad &S\leq L \\
V^n_{i,j}-(a^n_iV^{n}_{i-1,j}+b^n_iV^{n+1}_{i,j}+c^n_iV^{n}_{i+1,j})&=V^{n+1}_{i,j+1} \quad &S > L 
\end{aligned}
\right.
\end{equation}
其对应的三对角矩阵形式为
\begin{equation}
\left(
\begin{matrix}
1-2a^n_0-b^n_0 & a^n_0-c^n_0 & 0       &\ldots  & \ldots &\ldots     & 0 \\
-a^n_1       & 1-b^n_1   & -c^n_1    &\ldots  & \ldots &\ldots     & 0       \\
\vdots     & \ddots  & \ddots  &\ddots  & \vdots & \vdots   & \vdots   \\
0          & \ldots  & -a^n_I    & 1-b^n_I  & -c^n_I    & \ldots  & 0    \\
\vdots     & \vdots  & \vdots  &\ddots  & \ddots & \ddots   & \vdots   \\
0          & \ldots  & \ldots  &\ldots  & a^n_{M-1} & 1-b^n_{M-1} & -c^n_{M-1}   \\
0          & \ldots  & \ldots  &\ldots  &0        & c^n_M-a^n_M & 1-2c^n_M-b^n_M      
\end{matrix}
\right)
\left(
\begin{matrix}
V_{0,j}^n\\
V_{1,j}^n\\
\vdots\\
V_{I,j}^n\\
\vdots\\
V_{M-1,j}^n\\
V_{M,j}^n
\end{matrix}
\right)
=
\left(
\begin{matrix}
V_{0,j}^{n+1}\\
V_{1,j}^{n+1}\\
\vdots\\
V_{I,j}^{n+1}\\
\vdots\\
V_{M-1,j+1}^{n+1}\\
V_{M,j+1}^{n+1}
\end{matrix}
\right)
\end{equation}
\subsection{稳定性}
对于隐式差分方程\eqref{hcon1}和\eqref{hcon2},其增长因子为
\begin{equation}
G=\frac{1}{1-(a_ie^{-ikj\Delta h}+b_i+c_ie^{ikj\Delta h})}
\end{equation}
可推导出
\begin{equation}
\frac{1}{|G|}=1+r\Delta t+\frac{(r-q)S_i\Delta t}{2\Delta h}(1-\cos k\Delta h)+\frac{2\sigma^2S_i^2U_{n}\Delta t}{\Delta h^2}\sin^2\frac{kj\Delta h}{2}>1
\end{equation}
因此当$U_n>0$时,由于$1-\cos k\Delta h\geq 0$以及$\sin^2(kj\Delta h/2)>0$,因此恒有$|G|<1$,由vonNeuman准则知此隐式差分格式绝对稳定。即差分方程的稳定性不受网格剖分的影响。

\section{算例}
\subsection{布朗运动}
选取参数$L=12$,$D=0.1$,$E=10$,$T=1$,$\sigma=0.2$,$r=0.05$,$q=0$。当$W_t$为布朗运动时,即$U_t=1$,使用隐式差分法所得到的连续型和累计型向上敲出看涨巴黎期权价格分别如图\ref{paris2}和图\ref{parisc2}所示
\begin{figure}[H]
\begin{minipage}{0.48\linewidth}
\label{paris2}
\includegraphics[width=8.5cm]{code/paris2.jpg}
\caption{连续型向上敲出看涨巴黎期权($t=0.5$)}
\end{minipage}
\begin{minipage}{0.48\linewidth}
\label{parisc2}
\includegraphics[width=8.5cm]{code/parisc2.jpg}
\caption{累计型向上敲出看涨巴黎期权($t=0.5$)}
\end{minipage}
\end{figure}
图\ref{paris2}为连续型向上敲出巴黎期权的价格,对于任何的参数$\tau$,最高点都约出现在$S=10.7$处,且峰值恒定。峰值左边对任何参数$\tau$,巴黎期权的价格只受股价参数$S$的影响。而峰值右边,巴黎期权的价格随着计时器$\tau$的增大而具有更快的下降速度。

图\ref{parisc2}为累计型向上敲出巴黎期权的价格,峰值会随着计时器$\tau$的增大而减小,并且稍稍向股价$S$减少的的方向便宜。整个函数图像会随着计时器$\tau$的增大而变得更平缓。对比连续型和累计型向上巴黎期权的价格,同样参数下,由于累计型巴黎期权中计时器不清零,因此其敲出的可能性更大,从而价格比连续型巴黎期权的价格更低。
\iffalse
\subsection{分数布朗运动}
布朗运动为分数布朗运动的当$H=0.5$的特殊情况。下面考虑当$W_t$为分数布朗运动时,不同参数$H$,对巴黎期权价格的影响。使用隐式差分法所得到的连续型和累计型向上敲出看涨巴黎期权价格分别为
\begin{figure}[H]
\begin{minipage}{0.48\linewidth}
\includegraphics[width=8.5cm]{code/paris_fbm1.jpg}
\caption{连续型向上敲出看涨巴黎期权}
\end{minipage}
\begin{minipage}{0.48\linewidth}
\includegraphics[width=8.5cm]{code/parisc_fbm1.jpg}
\caption{累计型向上敲出看涨巴黎期权}
\end{minipage}
\end{figure}
与布朗运动相比,在分数布朗运动的模型中,连续型与累计型巴黎期权的价格都有处明显的区别。首先表现为峰值下降并向左移动,其次是两侧曲面的下滑速度减满慢。

\begin{figure}[H]
\begin{minipage}{0.48\linewidth}
\includegraphics[width=8.5cm]{code/paris_fbm2.jpg}
\caption{连续型向上敲出看涨巴黎期权}
\end{minipage}
\begin{minipage}{0.48\linewidth}
\includegraphics[width=8.5cm]{code/parisc_fbm2.jpg}
\caption{累计型向上敲出看涨巴黎期权}
\end{minipage}
\end{figure}
固定时间值$t$,同样能看到,相比布朗运动,分数布朗运动下巴黎期权价格的峰值向左移动,而相比连续型巴黎期权,累计型巴黎期权价格还随着计时器的增加而更快地减小。

考虑参数股价$S$在不同的参数$H$下对连续型向上敲出巴黎期权价格的影响
\begin{figure}[H]
\begin{center}
\label{paris_fbm}
\includegraphics[width=10cm]{code/paris_fbm.jpg}
\caption{$S$与$V$的关系图}
\end{center}
\end{figure}
由图\ref{paris_fbm}可以看出,随着分数布朗运动的参数$H$的增加,标的资产的长期依赖性增强。股票的历史价格对股票未来走势的预测性增强,未来的不确定性减弱,整个巴黎期权价格的图像就会由陡峭逐渐变得肥胖。
\fi

\subsection{更广泛的高斯过程}
方程\eqref{hcon1}和\eqref{hcon2}可写成
\begin{equation}
-a^n_iV^{n}_{i-1}+(1-b^n_i)V^{n}_i-c^n_iV^{n}_{i+1}=V^{n+1}_i
\end{equation}
以及
\begin{equation}
-a^n_iV^{n}_{i-1,j}+(1-b^n_i)V^{n}_{i,j}-c^n_iV^{n}_{i+1,j}=V^{n+1}_{i,j+1}
\end{equation}
显然不同的高斯过程,影响计算结果的只有方程中的参数$U_n$,即方差函数的导数$U_t$。为了更一般性的研究不同的高斯过程对巴黎期权价格的影响,考虑表\ref{t1}中的三个不同的$U_t$函数(显然$U_3$即为参数$H=0.7$的分数布朗运动),研究不同的高斯过程对巴黎期权价格的影响。

\begin{table}[H]
\begin{minipage}{0.48\textwidth}
\caption{算例使用的协方差函数}
\centering
\label{t1}
\begin{tabular}{c|c|c|c|c}
\hline
 &$\alpha$&$\gamma$&$F_t$ & $U_t$ \\
\hline
$U_1$&$1$&$1$&$t$ & $1$ \\
$U_2$&$0.7$&$1$ & $0.7t$ & $0.7$ \\
$U_3$&$1.4$&$0.4$ & $t^{1.4}$ & $1.4t^{0.4}$ \\
\hline
\end{tabular}
\end{minipage}
\begin{minipage}{0.48\textwidth}
\centering
\begin{figure}[H]
\label{ut}
\centering
\includegraphics[width=8cm]{code/ut.jpg}
\caption{$U_t$的函数图}
\end{figure}
\end{minipage}
\end{table}

如图\ref{ut}所示,当$t>0.43$的时,有$U_2<U_1<U_3$;当$0.18<t<0.42$的时候,$U_2<U_3<U_1$;当$t<1.7$时,$U_3<U_2<U_1$。对不同的高斯过程,利用追赶法求解方程\eqref{hcon1}和\eqref{hcon2},以及方程组\eqref{hcmu},可分别得到的连续型和累计型向上敲出看涨巴黎期权价格,分别如图\ref{mg2}和图\ref{mc2}所示。显然,在固定的时间$t$下,不同的高斯过程影响了图像的陡峭程度。但类似的,连续型巴黎期权在不同的参数$\tau$下,峰值都出现在同一股价处。峰值左边只受股价$S$的影响,峰值右边会随着计时器$\tau$的增大而具有更快的下滑。同样参数下,累计型巴黎期权的价格更低。并且峰值会随着计时器$\tau$的增大而减小,并左移。函数图像会随着计时器$\tau$的增大而平缓。

\begin{figure}[H]
\label{mg2}
\begin{minipage}{0.48\linewidth}
\includegraphics[width=8.5cm]{code/mg2.jpg}
\caption{连续型向上敲出看涨巴黎期权价格(t=0.5)}
\end{minipage}
\begin{minipage}{0.48\linewidth}
\label{mc2}
\includegraphics[width=8.5cm]{code/mc2.jpg}
\caption{累计型向上敲出看涨巴黎期权价格(t=0.5)}
\end{minipage}
\end{figure}

下面考虑不同的时间$t$下不同的高斯过程对巴黎期权价格的影响。图\ref{t0.5}与\ref{t0.5c}为当$t=0.5$的情形,$U_2$对应的图像最陡峭,峰值最高且偏右,而$U_3$对应的图像最平缓,峰值最低且偏左。
当$t=0.2$时,如图\ref{t0.2}与\ref{t0.2c},高斯过程$U_1$对应的图像最平缓,峰值也最低。当$t=0.2$时,由图\ref{t0}与\ref{t0c}可看出,连续型巴黎期权的图像中,最陡峭且峰值最高的图像为$U_3$,而累计型巴黎期权的图像中,最陡峭,峰值最高的图像依然为$U_2$。事实上,任何高斯过程对应的是相同的终端条件(式\eqref{hb})。数值计算的过程是从终端往前计算的
$$V^n\rightarrow V^{n-1} \rightarrow \ldots \rightarrow V^0$$
$U_n$的值越大,数值迭代计算会使图像会更快的趋于平缓,峰值更快的下降。而从高斯过程的性质来看,$U_t$越小,意味着方差函数增长缓慢,波动性越小,其盈亏的预测性更强,因此巴黎期权价格的函数图像就会更陡峭,峰值更高。

\begin{figure}[H]
\begin{minipage}{0.48\linewidth}
\label{t0.5}
\includegraphics[width=8.5cm]{code/t0.5.jpg}
\caption{连续型巴黎期权的价格($t=0.5$,$\tau=0.01$)}
\end{minipage}
\begin{minipage}{0.48\linewidth}
\label{t0.5c}
\includegraphics[width=8.5cm]{code/tc0.5.jpg}
\caption{累计型巴黎期权的价格($t=0.5$,$\tau=0.01$)}
\end{minipage}
\end{figure}

\begin{figure}[H]
\begin{minipage}{0.48\linewidth}
\label{t0.2}
\includegraphics[width=8.5cm]{code/t0.2.jpg}
\caption{连续型巴黎期权的价格($t=0.2$,$\tau=0.01$)}
\end{minipage}
\begin{minipage}{0.48\linewidth}
\label{t0.2c}
\includegraphics[width=8.5cm]{code/tc0.2.jpg}
\caption{累计型巴黎期权的价格($t=0.2$,$\tau=0.01$)}
\end{minipage}
\end{figure}

\begin{figure}[H]
\begin{minipage}{0.48\linewidth}
\label{t0}
\includegraphics[width=8.5cm]{code/t0.jpg}
\caption{连续型巴黎期权的价格($t=0$,$\tau=0.01$)}
\end{minipage}
\begin{minipage}{0.48\linewidth}
\label{t0c}
\includegraphics[width=8.5cm]{code/tc0.jpg}
\caption{累计型巴黎期权的价格($t=0$,$\tau=0.01$)}
\end{minipage}
\end{figure}

\section{结束语}
对金融衍生品的定价研究,往往需假设金融资产的运动模型。而最常见的几位即为布朗运动。许多学者也基于布朗运动的假设建立了各类巴黎期权的定价模型。然而常见的金融资产的运动模型还包括分数布朗运动,次分数布朗运动等等各种具有记忆性,相似性的运动模型。各种研究也表明,在某些金融衍生品的定价中,这些具有记忆性和相似性的运动模型更能反映真实金融资产的运动特征。因此,在巴黎期权的定价模型中,将布朗运动的模型假设推广到更一般的高斯过程便是本文研究的初衷。之后在对其他巴黎期权或者可转债等金融衍生品进行定价的时候,可选择最逼近实际资产运动的高斯过程,再对其金融衍生品进行定价求解,由此可以得到更高的准确度。

但是当高斯过程$W_t$既不是Markovian过程,也不是半鞅时,按照路径型积分建立的金融市场数学模型是存在套利机会的。因此本文通过引入Wick-Ito乘积,构造一般高斯过程下,巴黎期权价值需满足的随机偏微分方程。由于微分方程的求解中,隐式差分法往往具有更高的稳定性,因此本文直接构建无条件稳定的隐式差分格式对巴黎期权的偏微分方程进行求解。但是隐式差分法往往也伴随着更庞大的计算量,为此本文通过使用追赶法法进行方程组的求解。由于追赶法不需要求出矩阵的逆,因此可以将计算量大大的减小。

文章的最后使用算例分别求解连续型和累计型巴黎期权的价格,并讨论各个参数巴黎期权的影响。在任意时刻$t$中,连续型巴黎期权在不同的计时器$\tau$下,峰值都出现在同一股价处。峰值左边只受股价$S$的影响,与$\tau$无关,峰值右边会随着计时器$\tau$的增大而更快的下滑。而同样参数下,累计型巴黎期权由于计时器不清零,因此比连续型的价格更低。并且峰值会随着计时器$\tau$的增大而减小,并稍稍左移。整个函数图像会随着计时器$\tau$的增大而趋于平缓。不同的高斯过程,影响巴黎期权计算结果的只有高斯过程的方差函数的导数$U_t$。$U_t$越小,意味着方差函数增长缓慢,波动性越小,其盈亏的预测性更强,因此连续型与累计型巴黎期权价格的函数图像就会更陡峭,峰值更高。相反的,当$U_t$增大时,巴黎期权价格的函数图像则会更变得更平缓,峰值也会降低。


\begin{thebibliography}{10}
\bibitem{ref1}Chesney M, Cornwall J, Jeanblanc M, Kentwell G, Yor M. Parisian pricing [J]. Risk, 1997, 10(1): 77~79.
\bibitem{ref2}宋斌, 林则夫, 刘黎黎, 张冰洁. 基于博弈期权的可转债定价模型及其实证研究 [J]. 系统管理学报, 2013, 22(6): 758~767.
\bibitem{ref3}Haber J, Schonbucher J, Wilmott P. Pricing Parisian options [J]. The Journal of Theoretical and Applied Finance, 1999, 6(3): 71~79.
\bibitem{ref4}罗俊, 吴雄华. 巴黎期权定价问题的数值方法 [J]. 数值计算与计算机应用, 2004, 25(2): 81~89.
\bibitem{ref5}宋斌, 周湛满, 魏琳, 张冰洁. 巴黎期权的PDE定价及隐性差分法研究 [J]. 系统工程学报, 2013, 28(6): 764~774.
\bibitem{ref6}wok K, Lau W. Pricing algorithms for options with exotic path dependence [J]. Journal of Derivatives, 2001, 9(1): 28~38.
\bibitem{ref7}Longstaff F A, Schwartz E S. Valuing American options by simulation: A simple least-square approach [J]. Review of Financial Studies, 2001, 14(1): 113~147.
\bibitem{ref8}宋斌, 井帅. 美式巴黎期权的定价模型与数值方法 [J]. 系统工程, 2015, 33(2): 1~8.
\bibitem{ref9}wok K, Lau W. Pricing algorithms for options with exotic path dependence [J]. Journal of Derivatives, 2001, 9(1): 28~38.
\bibitem{ref10}Longstaff F A, Schwartz E S. Valuing American options by simulation: A simple least-square approach [J]. Review of Financial Studies, 2001, 14(1): 113~147.
\bibitem{ref11}宋斌, 井帅. 美式巴黎期权的定价模型与数值方法 [J]. 系统工程, 2015, 33(2): 1~8.
\bibitem{ref12} Rogers L. Arbitrage with fractional Brownian motion[J]. Mathematical Finance, 1997, 7(1): 95~105.
\bibitem{ref13} Nualart D, Taqqu M S. Wick-Ito formula for regular processes and applications to Black and Scholes formula [J]. Stochastics: An International Journal of Probability and Stochastic Processes, 2008, 80(5): 477~487.
\bibitem{ref14} Dai Q, Singleton, K J. Specification analysis of affine term structure models [J].Journal of Finance, 2000, 55(5): 1943~1978.
\bibitem{ref15} Peters E F. Fractal structure in the capital markets [J].Financial analyst Journal, 1989, 7(32): 434~453.
\bibitem{ref16} Hu Y, Øksendal B. Fractional white noise calculus and application to finance [J]. Infinite Dimensional Analysis, Quantum Probability and Related Topics, 2003, 1(6):1~32.
\bibitem{ref17}Lei P, Nualart D. A decomposition of the bi-fractional Brownian motion and some applications [J]. Statistics and Probability Letters, 2009, 79(5): 619~624.
\bibitem{ref18} Tudor C. Inner product spaces of integrands associated to sub-fractional Brownian motion [J]. Statistics and Probability Letters, 2008, 78(14): 2201~2209.
\bibitem{ref19} Zili M. On the mixed fractional Brownian motion[J]. Journal of Applied Mathematics and Stochastic Analysis, 2006, (32435):1~9.


\end{thebibliography}


\end{document}
